\documentclass[a4j,twoside,openright,11pt]{jarticle}
%
\usepackage{amsmath,amssymb}
\usepackage{bm}
\usepackage{graphicx}
\usepackage{ascmac}
\usepackage{listliketab}
\usepackage{url}
\usepackage{listings}
\usepackage{color}

\setlength{\textwidth}{15.92cm}
\setlength{\oddsidemargin}{0mm}
\setlength{\evensidemargin}{0mm}
\setlength{\topmargin}{-1cm}
\setlength{\textheight}{23.5cm}
\setlength{\footskip}{18mm}

%
\pagestyle{plain}
\begin{document}

\begin{screen}
\huge
\begin{center}
{\bf 流体力学 第1回}\\
\end{center}

\normalsize
\begin{flushright}
九州工業大学 機械知能工学科 機械知能コース  坂本 悠作\\連絡先:n104069y@mail.kyutech.jp \hspace{0.2in} 2015年4月8日
\end{flushright}
\end{screen}

\section{流体の性質}
\subsection{流れの分類}
流れの分類には、次のような分類方法が一般的である。
\begin{itemize}
\item 非圧縮性流体(密度が一様)
\begin{itemize}
\item 非粘性流体
\item 粘性流体
\end{itemize}

\item 圧縮性流体(密度が一様ではない)
\begin{itemize}
\item 非粘性流体
\item 粘性流体
\end{itemize}
\end{itemize}
\subsection{定常流れと非定常流れ}
定常流れとは、\textcolor{red}{「流れの状態が時間的に変化しないもの」}非定常流れとは、\textcolor{red}{「流れの状態が時間的に変化するもの」}を指します。例えば、機械工学実験で行ったサイホン管の実験は、水を使っているので非圧縮性流体ですが、圧力と速度が徐々に変化していく様子を測定しました。実験では定常流れとして扱いましたが、それはごく短い時間の話で、長いスパンで見ればこれは非定常流れと取れます。河川を流れる水も、雨の日には水量、流速が増えます。永久に定常流れとして見ることができるのは、人工物だけのような気がします。
\subsection{2次元流れと3次元流れ}
\begin{itemize}
\item 2次元流れ\\
x,yの直交座標を取れば、十分流れの状態を表せる流れ。パイプなど。
\item 3次元流れ\\
3次元座標でないと表せない流れ。こちらのほうが、どのような流れにも対応できる、汎用性の有るものではあるが、より複雑になる。
\end{itemize}
\subsection{流れを表す量}
\begin{enumerate}
\item 座標系\\
\begin{enumerate}
\item 直交座標
\item 円柱座標
\item 球座標
\end{enumerate}
\item 速度\\
速度の記述は、以下のように記述する。
{\bf V}=(u,v,w)とかく。
\item 圧力\\
\begin{eqnarray}
p=\lim_{\Delta A \rightarrow0}\frac{\Delta P}{\Delta A}\\
\nonumber\\
\tau=\lim_{\Delta A \rightarrow 0}\frac{\Delta T}{\Delta A}
\end{eqnarray}
\item 流れを表す線\\
\begin{enumerate}
\item 流脈線(streak line)\\
空間の特定の点を通過すた、流体のつながりとしてできる点
\item 流跡線(path line)\\
流体の塊がたどる軌跡
\item 流線(stream line)\\
線の接線がその速度の方向と一致する曲線
\end{enumerate}

\end{enumerate}

\section{流れ場の未知量と方程式}
u,v,w,p,$\rho$,のパラメータは、x,y,z,tの関数として扱われる。
\end{document}