\chapter{軸受け}
\section{軸受けにかかる力のまとめ}
\begin{table}[htb]
\begin{center}
  \caption{表題}
  \begin{tabular}{|l||c|c|c|c|} \hline
軸受け番号& 最小軸径[mm] &ラジアル荷重Fr[N] &スラスト荷重Fa[N]&回転数[rpm]\\\hline
1&20&972.6&0&1300\\
2&44&1784.21&972.87&1300\\
3&30&6970.86&0&328.5714\\
4&26&4477&1978.13&328.5714\\
5&82&5255.86&2951&108.0235\\
6&33&4479.24&0&108.0235\\
\hline
  \end{tabular}
\end{center}
\end{table}





\section{軸受け計算}
\newpage
\subsection{軸受け1の選定}
\subsubsection{軸受け1データ}
\begin{table}[htb]
\begin{center}
  \caption{軸受け1データ}
  \begin{tabular}{lll} \hline
名称&記号&値\\
ラジアル荷重&$F_r$&972.6[N]\\
スラスト荷重&$F_a$&0\\
回転数&n&1300[rpm]\\
定格寿命&$L_h$&20000以上 [hour]\\
最小軸径&$\alpha$&20 [mm]\\
軸受け種類&p(玉軸受け)&3[$\cdot$]\\
\hline
  \end{tabular}
\end{center}
\end{table}

\begin{table}[htb]
\begin{center}
  \caption{NSK60/28}
  \begin{tabular}{lll} \hline
名称&記号&値\\
内径& d &28 [mm]\\
外径& D &52 [mm]\\
基本動定格荷重&$C_{r}$&12500\\
基本静定格荷重&$C_{0r}$&7400\\
軸受各部の形状および適用する&$f_0$&14.5\\
応力水準によって定まる係数&&\\
\hline
  \end{tabular}
\end{center}
\end{table}


\subsubsection{軸受け1検討}
\begin{eqnarray}
寿命係数f_h &=& \left( \frac{L_h}{500} \right)^{1/p} = \left( \frac{20000}{500} \right)^{1/3} = 3.420\\
速度係数f_n &=& \left( \frac{100}{3n} \right)^{1/p} = \left( \frac{100}{3 \times 1300} \right)^{1/3} = 0.29488\\
P &=& XF_r+YF_a = 972.6\\
C &=& \frac{f_h}{f_n} \times P = 11280.1546[N]
\end{eqnarray}

\subsubsection{軸受け1再検討}
アキシアル荷重が働いていないので、自動的にX=1,Y=0とする。
\begin{eqnarray}
寿命時間L_h &=& 500{f_n}^p(C_r/P)^p\\
           &=& 500 \times \frac{100}{3 \times 1300} \times (12500/972.6)^3\\
           &=& 27126.523 \geq 20000\\
寿命係数f_h &=& \left( \frac{L_h}{500} \right)^{1/p} = \left( \frac{27126.523}{500} \right)^{1/3} = 3.786
\end{eqnarray}

\subsubsection{静荷重の確認}
\begin{eqnarray}
0.6F_r+0.5F_a=0.6 \times 972.6 + 0.5 \times 0 = 486.3 \leq F_r\\
よって、静等価荷重P_0 = F_r=972.6\\
f_s = \frac{C_{0r}}{P_0} = \frac{7400}{972.6}\geq 1
\end{eqnarray}

\newpage
\subsection{軸受け2の選定}
\subsubsection{軸受け2データ}
\begin{table}[htb]
\begin{center}
  \caption{軸受け2データ}
  \begin{tabular}{lll} \hline
名称&記号&値\\
ラジアル荷重&$F_r$&1784.21[N]\\
スラスト荷重&$F_a$&972.87[N]\\
回転数&n&328.5714[rpm]\\
定格寿命&$L_h$&20000以上 [hour]\\
最小軸径&$\alpha$&44 [mm]\\
軸受け種類&p(玉軸受け)&3[$\cdot$]\\
\hline
  \end{tabular}
\end{center}
\end{table}

\begin{table}[htb]
\begin{center}
  \caption{NSK6013}
  \begin{tabular}{lll} \hline
名称&記号&値\\
内径& d &65 [mm]\\
外径& D &100 [mm]\\
基本動定格荷重&$C_{r}$&30500\\
基本静定格荷重&$C_{0r}$&25200\\
軸受各部の形状および適用する&$f_0$&15.8\\
応力水準によって定まる係数&&\\
\hline
  \end{tabular}
\end{center}
\end{table}


\subsubsection{軸受け2検討}
X=0.56,Y=1.00とする。
\begin{eqnarray}
寿命係数f_h &=& \left( \frac{L_h}{500} \right)^{1/p} = \left( \frac{20000}{500} \right)^{1/3} = 3.420\\
速度係数f_n &=& \left( \frac{100}{3n} \right)^{1/p} = \left( \frac{100}{3 \times 1300} \right)^{1/3} = 0.29488\\
\frac{F_a}{F_r} &=& \frac{972.87}{1784.21} = 0.5453(\geq 0.44)\\
P &=& XF_r+YF_a = 0.56 \times 1784.21 + 1.00 \times 972.87 = 1972.0276\\
C &=& \frac{f_h}{f_n} \times P = 11.59794 \times 1972.0276 = 22941.045[N]
\end{eqnarray}

\subsubsection{軸受け2再検討}
\begin{eqnarray}
寿命時間L_h &=& 500{f_n}^p(C_r/P)^p\\
           &=& 500 \times \frac{100}{3 \times 1300} \times (30500/1972.0276)^3\\
           &=& 47431.4 \geq 20000
\end{eqnarray}
X=0.56,Y=1.77とする。
\begin{eqnarray}
P &=& XF_r+YF_a = 0.56 \times 1784.21 + 1.77 \times 972.87 = 2721.14\\
寿命時間L_h &=& 500{f_n}^p(C_r/P)^p\\
           &=& 500 \times \frac{100}{3 \times 1300} \times (30500/2721.14)^3\\
           &=& 138194.90 \geq 20000\\
寿命係数f_h &=& \left( \frac{L_h}{500} \right)^{1/p} = \left( \frac{138194.90}{500} \right)^{1/3} = 6.514
\end{eqnarray}

\subsubsection{静荷重の確認}
\begin{eqnarray}
P_0&=&0.6F_r+0.5F_a\\
&=&0.6 \times 1784.21 + 0.5 \times 972.87 = 1556.961 \leq F_r\\
よって、静等価荷重P_0 &=& F_r =1784.21\\
f_s &=& \frac{C_{0r}}{P_0} = \frac{44500}{1784.21}\geq 1
\end{eqnarray}











\newpage
\subsection{軸受け3の選定}

\subsubsection{軸受け3データ}
\begin{table}[htb]
\begin{center}
  \caption{軸受け3データ}
  \begin{tabular}{lll} \hline
名称&記号&値\\
ラジアル荷重&$F_r$&6970.86[N]\\
スラスト荷重&$F_a$&0\\
回転数&n&328.5714[rpm]\\
定格寿命&$L_h$&20000以上 [hour]\\
最小軸径&$\alpha$&30 [mm]\\
軸受け種類&p(玉軸受け)&3[$\cdot$]\\
\hline
  \end{tabular}
\end{center}
\end{table}

\begin{table}[htb]
\begin{center}
  \caption{NSK6309}
  \begin{tabular}{lll} \hline
名称&記号&値\\
内径& d &45 [mm]\\
外径& D &100 [mm]\\
基本動定格荷重&$C_{r}$&53000\\
基本静定格荷重&$C_{0r}$&32000\\
軸受各部の形状および適用する&$f_0$&13.1\\
応力水準によって定まる係数&&\\
\hline
  \end{tabular}
\end{center}
\end{table}

\subsubsection{軸受け3検討}
\begin{eqnarray}
寿命係数f_h &=& \left( \frac{L_h}{500} \right)^{1/p} = \left( \frac{20000}{500} \right)^{1/3} = 3.420\\
速度係数f_n &=& \left( \frac{100}{3n} \right)^{1/p} = \left( \frac{100}{3 \times 328.5714} \right)^{1/3} = 0.4664\\
P &=& XF_r+YF_a = 6970.86\\
C &=& \frac{f_h}{f_n} \times P = 51115.654[N]
\end{eqnarray}

\subsubsection{軸受け3再検討}
アキシアル荷重が働いていないので、自動的にX=1,Y=0とする。
\begin{eqnarray}
寿命時間L_h &=& 500{f_n}^p(C_r/P)^p\\
           &=& 500 \times \frac{100}{3 \times 328.5714} \times (53000/6970.86)^3\\
           &=& 22293.975 \geq 20000\\
寿命係数f_h &=& \left( \frac{L_h}{500} \right)^{1/p} = \left( \frac{22293.975}{500} \right)^{1/3} = 3.546
\end{eqnarray}

\subsubsection{静荷重の確認}
\begin{eqnarray}
0.6F_r+0.5F_a=0.6 \times 6970.86 + 0.5 \times 0 = 4182.516 \leq F_r\\
よって、静等価荷重P_0 = F_r = 6970.86\\
f_s = \frac{C_{0r}}{P_0} = \frac{32000}{6970.86} \geq 1
\end{eqnarray}







\newpage
\subsection{軸受け4の選定}
\subsubsection{軸受け4データ}
\begin{table}[htb]
\begin{center}
  \caption{軸受け4データ}
  \begin{tabular}{lll} \hline
名称&記号&値\\
ラジアル荷重&$F_r$&4477[N]\\
スラスト荷重&$F_a$&1978.13[N]\\
回転数&n&328.5714[rpm]\\
定格寿命&$L_h$&20000以上 [hour]\\
最小軸径&$\alpha$&26 [mm]\\
軸受け種類&p(玉軸受け)&3[$\cdot$]\\
\hline
  \end{tabular}
\end{center}
\end{table}

\begin{table}[htb]
\begin{center}
  \caption{NSK6309}
  \begin{tabular}{lll} \hline
名称&記号&値\\
内径& d &45 [mm]\\
外径& D &100 [mm]\\
基本動定格荷重&$C_{r}$&53000\\
基本静定格荷重&$C_{0r}$&32000\\
軸受各部の形状および適用する&$f_0$&13.1\\
応力水準によって定まる係数&&\\
\hline
  \end{tabular}
\end{center}
\end{table}

\subsubsection{軸受け4検討}
X=0.56,Y=1.00とする。
\begin{eqnarray}
寿命係数f_h &=& \left( \frac{L_h}{500} \right)^{1/p} = \left( \frac{20000}{500} \right)^{1/3} = 3.420\\
速度係数f_n &=& \left( \frac{100}{3n} \right)^{1/p} = \left( \frac{100}{3 \times 328.5714} \right)^{1/3} = 0.4664\\
\frac{F_a}{F_r} &=& \frac{1978.13}{4477} = 0.4418(\geq 0.44)\\
P &=& XF_r+YF_a = 0.56 \times 4477 + 1.00 \times 1978.13 = 4485.25\\
C &=& \frac{f_h}{f_n} \times P = 7.33276 \times 4485.25 = 32889.26[N]
\end{eqnarray}

\subsubsection{軸受け4再検討}
\begin{eqnarray}
寿命時間L_h &=& 500{f_n}^p(C_r/P)^p\\
           &=& 500 \times \frac{100}{3 \times 328.5714} \times (53000/4485.25)^3\\
           &=& 83692.52 \geq 20000
\end{eqnarray}

X=0.56,Y=1.71とする。
\begin{eqnarray}
P &=& XF_r+YF_a = 0.56 \times 4477 + 1.71 \times 1978.13 = 5889.72\\
寿命時間L_h &=& 500{f_n}^p(C_r/P)^p\\
           &=& 500 \times \frac{100}{3 \times 328.5714} \times (40500/5889.27)^3\\
           &=& 36971.09 \geq 20000\\
寿命係数f_h &=& \left( \frac{L_h}{500} \right)^{1/p} = \left( \frac{36971.09}{500} \right)^{1/3} = 4.197
\end{eqnarray}

\subsubsection{静荷重の確認}
\begin{eqnarray}
P_0&=&0.6F_r+0.5F_a\\
&=&0.6 \times 4477 + 0.5 \times 1978.13 = 3675.265 \leq F_r\\
よって、静等価荷重P_0 &=& F_r = 4477\\
f_s &=& \frac{C_{0r}}{P_0} = \frac{32000}{4477}\geq 1
\end{eqnarray}


\newpage
\subsection{軸受け5の選定}
\subsubsection{軸受け5データ}
\begin{table}[htb]
\begin{center}
  \caption{軸受け5データ}
  \begin{tabular}{lll} \hline
名称&記号&値\\
ラジアル荷重&$F_r$&5255.86[N]\\
スラスト荷重&$F_a$&2951[N]\\
回転数&n&108.0235[rpm]\\
定格寿命&$L_h$&20000以上 [hour]\\
最小軸径&$\alpha$&82 [mm]\\
軸受け種類&p(玉軸受け)&3[$\cdot$]\\
\hline
  \end{tabular}
\end{center}
\end{table}

\begin{table}[htb]
\begin{center}
  \caption{NSK6018}
  \begin{tabular}{lll} \hline
名称&記号&値\\
内径& d & 90 [mm]\\
外径& D & 140 [mm]\\
基本動定格荷重&$C_{r}$&58000\\
基本静定格荷重&$C_{0r}$&50000\\
軸受各部の形状および適用する&$f_0$&15.6\\
応力水準によって定まる係数&&\\
\hline
  \end{tabular}
\end{center}
\end{table}

\subsubsection{軸受け5検討}
X=0.56,Y=1.00とする。
\begin{eqnarray}
寿命係数f_h &=& \left( \frac{L_h}{500} \right)^{1/p} = \left( \frac{20000}{500} \right)^{1/3} = 3.420\\
速度係数f_n &=& \left( \frac{100}{3n} \right)^{1/p} = \left( \frac{100}{3 \times 108.0235} \right)^{1/3} = 0.67575\\
\frac{F_a}{F_r} &=& \frac{2986}{5255.86} = 0.568(\geq 0.44)\\
P &=& XF_r+YF_a = 0.56 \times 5255.86 + 1.00 \times 2951 = 5894.2816\\
C &=& \frac{f_h}{f_n} \times P = 5.0610 \times 5894.2816 = 29830.959[N]
\end{eqnarray}

\subsubsection{軸受け5再検討}
\begin{eqnarray}
寿命時間L_h &=& 500{f_n}^p(C_r/P)^p\\
           &=& 500 \times \frac{100}{3 \times 108.0235} \times (147022. 60/5894.2816)^3\\
           &=& 147022.60 \geq 20000
\end{eqnarray}
X=0.56,Y=1.45とする。
\begin{eqnarray}
P &=& XF_r+YF_a = 0.56 \times 5255.86 + 1.71 \times 2951 =7989.49\\
寿命時間L_h &=& 500{f_n}^p(C_r/P)^p\\
           &=& 500 \times \frac{100}{3 \times 108.0235} \times (58000/7989.49)^3\\
           &=& 59027.90 \geq 20000\\
寿命係数f_h &=& \left( \frac{L_h}{500} \right)^{1/p} = \left( \frac{59027.90}{500} \right)^{1/3} = 4.9056
\end{eqnarray}
\subsubsection{静荷重の確認}
\begin{eqnarray}
P_0&=&0.6F_r+0.5F_a\\
&=&0.6 \times 5255.86 + 0.5 \times 2951 = 4629.016 \leq F_r\\
よって、静等価荷重P_0 &=& F_r = 5255.86\\
f_s &=& \frac{C_{0r}}{P_0} = \frac{50000}{5255.86}\geq 1
\end{eqnarray}


\newpage
\subsection{軸受け6の選定}

\subsubsection{軸受け6データ}
\begin{table}[htb]
\begin{center}
  \caption{軸受け6データ}
  \begin{tabular}{lll} \hline
名称&記号&値\\
ラジアル荷重&$F_r$&4479.24[N]\\
スラスト荷重&$F_a$&0\\
回転数&n&108.0235[rpm]\\
定格寿命&$L_h$&20000以上 [hour]\\
最小軸径&$\alpha$&33 [mm]\\
軸受け種類&p(玉軸受け)&3[$\cdot$]\\
\hline
  \end{tabular}
\end{center}
\end{table}

\begin{table}[htb]
\begin{center}
  \caption{NSK6310}
  \begin{tabular}{lll} \hline
名称&記号&値\\
内径& d &50 [mm]\\
外径& D &110 [mm]\\
基本動定格荷重&$C_{r}$&62000\\
基本静定格荷重&$C_{0r}$&38500\\
軸受各部の形状および適用する&$f_0$&13.2\\
応力水準によって定まる係数&&\\
\hline
  \end{tabular}
\end{center}
\end{table}


\subsubsection{軸受け6検討}
\begin{eqnarray}
寿命係数f_h &=& \left( \frac{L_h}{500} \right)^{1/p} = \left( \frac{20000}{500} \right)^{1/3} = 3.420\\
速度係数f_n &=& \left( \frac{100}{3n} \right)^{1/p} = \left( \frac{100}{3 \times 108.0235} \right)^{1/3} = 0.67575\\
P &=& XF_r+YF_a = 4479.24\\
C &=& \frac{f_h}{f_n} \times P = 22669.62752[N]
\end{eqnarray}

\subsubsection{軸受け6再検討}
アキシアル荷重が働いていないので、自動的にX=1,Y=0とする。
\begin{eqnarray}
寿命時間L_h &=& 500{f_n}^p(C_r/P)^p\\
           &=& 500 \times \frac{100}{3 \times 108.0235} \times (62000/4479.24)^3\\
           &=& 30257.740 \geq 20000\\
寿命係数f_h &=& \left( \frac{L_h}{500} \right)^{1/p} = \left( \frac{30257.74}{500} \right)^{1/3} = 3.926
\end{eqnarray}

\subsubsection{静荷重の確認}
\begin{eqnarray}
0.6F_r+0.5F_a=0.6 \times 4479.24 + 0.5 \times 0 = 2687.544 \leq F_r\\
よって、静等価荷重P_0 = F_r = 4479.24\\
f_s = \frac{C_{0r}}{P_0} = \frac{38500}{4479.24} \geq 1
\end{eqnarray}

\section{オイルシールの選定}
\subsection{軸受け2側オイルシール}
\begin{table}[htb]
\begin{center}
  \caption{商品コード:AA213600}
  \begin{tabular}{ll}
    \hline
    メーカー&NOK\\
    型式&TCJ\\
    内径&25\\
    外形&45\\
    厚さ&11\\
    材質&ニトリルゴム+PTFE焼付\\
    \hline
  \end{tabular}
\end{center}
\end{table}
\subsection{軸受け5側オイルシール}
\begin{table}[htb]
\begin{center}
  \caption{商品コード:AA213607}
  \begin{tabular}{ll}
    \hline
    商品コード&AA213607\\
    メーカーコード&GJ2651-P0\\
    メーカー&NOK\\
    型式&TCJ\\
    内径(mm)&45
    \\外径(mm)&62
    \\厚さ(mm)&9\\
    材質&ニトリルゴム+PTFE焼付\\
    \hline
  \end{tabular}
\end{center}
\end{table}

