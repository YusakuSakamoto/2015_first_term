\documentclass[a4j,twoside,openright,11pt]{jreport}
%
\usepackage{amsmath,amssymb}
\usepackage{bm}
\usepackage{graphicx}
\usepackage{ascmac}
\usepackage{listliketab}
\usepackage{url}
\usepackage{listings}

\setlength{\textwidth}{15.92cm}
\setlength{\oddsidemargin}{0mm}
\setlength{\evensidemargin}{0mm}
\setlength{\topmargin}{-1cm}
\setlength{\textheight}{23.5cm}
\setlength{\footskip}{18mm}

%
\pagestyle{plain}
\begin{document}

\begin{screen}
\huge
\begin{center}
{\bf 数値解析法 第1回課題}\\
\end{center}

\normalsize
\begin{flushright}
九州工業大学 機械知能工学科 機械知能コース  坂本 悠作\\連絡先:n104069y@mail.kyutech.jp \hspace{0.2in}提出日 2015年4月14日
\end{flushright}
\end{screen}

\section{問題1}

$f(x) = x^{3}-2x^{2}-x+2 =0$
の他の2解を、適当に区間を定めて、2分法により求めよ\\
\par
解答\\
収束判定条件\\
\begin{equation}
|f(x_i)|<\epsilon\nonumber
\end{equation}
ただし、$\epsilon$=0.05とする\\
\begin{table}[htb]
\begin{center}
  \caption{区間[a,b]=[-1.2,-0.9]}
  \begin{tabular}{|l||c|c|c|c|c|} \hline
i&$x_l$&$x_i$   &$x_r$ & $x_r-x_l$&$f(x_i)$\\\hline
1&-1.2 &-1.05  &-0.9 & 0.3      &-0.312625\\
2&-1.05&-0.975 &-0.9 & 0.15     &0.14689\\
3&-1.05&-1.0125&-0.975 & 0.075  &-0.075783\\
4&-1.0125&-0.99375&-0.975&0.0375&0.03730\\\hline
  \end{tabular}
\end{center}

\end{table}
\begin{table}[htb]
\begin{center}
  \caption{区間[a,b]=[1.9,2.2]}
  \begin{tabular}{|l||c|c|c|c|c|} \hline
i&$x_l$&$x_i$   &$x_r$ & $x_r-x_l$&$f(x_i)$\\\hline
1&1.9 &2.05  &2.2 & 0.3      &0.160125\\
2&1.9 &1.975 &2.05 & 0.15     &-0.072517\\
3&1.975&2.0125&2.05& 0.075 & 0.03812\\\hline
  \end{tabular}
\end{center}
\end{table}

よって、その他の解は-0.99375,2.0125
\section{問題2}
$\sqrt[3]{7}を、Newton$--$Rophson法を用いて求めよ$

解
\begin{eqnarray}
f(x)&=&x^3-7\nonumber\\
&=&0\nonumber
\end{eqnarray}
上のようにf(x)を設定して、xについて解く。

\begin{table}[htb]
\begin{center}
  \caption{初期値を$x_0$=2とする}
  \begin{tabular}{|l||c|c|c|} \hline
i&$x_i$&$f(x_i)$&$f(x_i)'$\\\hline
0&2    &1      &$3\times 2^2$\\
1&1.9166&0.04108&3$\times 1.9166^2$\\
2&1.9129&-3.423$\times 10^{-4}$ & $3\times 1.9129^2$\\\hline
  \end{tabular}
\end{center}
\end{table}

十分小さい値まで収束したので、これを答えとする。\\
解は、1.9129

\section{問題3}
超越方程式$f(x)=x^2-\cos{x}=0$の解を求めよ

Newton-Rophson法を用いる。

\begin{table}[htb]
\begin{center}
  \caption{初期値を$x_0$=2とする}
  \begin{tabular}{|l||c|c|c|} \hline
i&$x_i$ &$f(x_i)$&$f(x_i)'$\\\hline
0&2     &3.000  &$2\times 2 +\sin 2 = 4.035$\\
1&1.2565&0.5790 &$2\times 1.2565 + \sin 1.2565 = 2.5349$\\
2&1.0281&0.05715&$2\times 1.0281 + \sin 1.0281 = 2.07414$\\
3&1.0005&0.001153&$2\times 1.0005 + \sin 1.0005 = 2.01846$\\\hline
  \end{tabular}
\end{center}
\end{table}

$x^2-\cos x$は偶関数であるので、y軸に対して対象な挙動を示すので、解は、$\pm$1.0005

\end{document}