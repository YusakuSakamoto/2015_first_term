\documentclass[a4j,twoside,openright,11pt]{jarticle}
%
\usepackage{amsmath,amssymb}
\usepackage{bm}
\usepackage{graphicx}
\usepackage{ascmac}
\usepackage{listliketab}
\usepackage{url}
\usepackage{listings}

\setlength{\textwidth}{15.92cm}
\setlength{\oddsidemargin}{0mm}
\setlength{\evensidemargin}{0mm}
\setlength{\topmargin}{-1cm}
\setlength{\textheight}{23.5cm}
\setlength{\footskip}{18mm}

%
\pagestyle{plain}
\begin{document}

\begin{screen}
\huge
\begin{center}
{\bf 数値解析法 中間テスト対策}\\
\end{center}

\normalsize
\begin{flushright}
九州工業大学 機械知能工学科 機械知能コース 3年 坂本 悠作\\学籍番号 13104069 \hspace{0.2in}提出日 2015年6月15日
\end{flushright}
\end{screen}

\section{問題1}
$\frac{\partial u}{\partial x}$と、$\frac{\partial^2 u}{\partial x^2}$を、Taylor展開から始めて、2次の中心差分(中央差分)式で表すと共に、それぞれの打ち切り誤差を評価せよ。

\section{問題2}
ポテンシャル流れの流れ関数$\psi$はラプラスの方程式(次式)を満足する。
\begin{eqnarray}
\frac{\partial^2 \psi}{\partial x^2} + \frac{\partial^2 \psi}{\partial y^2} = 0
\end{eqnarray}

\subsection{上式を満足するように流れ場の全点で$\psi$の値を求めるためには、繰り返し計算によって、解を収束させることが必要になる理由を説明せよ。}
\subsection{緩和法について、フローチャートを描き、式を用いて説明せよ}
\subsection{「ステップを超える一様流れ」を例に、境界条件の与え方など、実際の解法について説明せよ}

\section{問題3}
1次元の伝熱方程式:$\frac{\partial T}{\partial t} = k \frac{\partial^2 T}{\partial x^2}$を例にして、クランク・ニコルソン法を用いた解法について説明せよ

\section{問題4}
次の語句を説明せよ
\subsection{前進差分と後退差分}
Taylor展開によって得られる前進差分の式を次に示す。前進差分の場合は、ある点(i,j)からみて(i+1,j)の点がどうなっているかを表したものである。
\begin{eqnarray}
u(x+\delta x) = u(x) + \frac{\partial u}{\partial x}\delta x + \frac{1}{2!}\frac{\partial^2 u}{\partial x^2}\delta x^2 + \frac{1}{3!}\frac{\partial^3 u}{\partial x^3}\delta x^3 + \cdots
\end{eqnarray}
後退差分の場合は、ある点(i,j)からみて(i-1,j)の点がどうなっているかを表したものである。
\begin{eqnarray}
u(x-\delta x) = u(x) - \frac{\partial u}{\partial x}\delta x + \frac{1}{2!}\frac{\partial^2 u}{\partial x^2}\delta x^2 - \frac{1}{3!}\frac{\partial^3 u}{\partial x^3}\delta x^3 + \cdots
\end{eqnarray}

\subsection{高次差分式}
上の前進差分と後退差分のときにはテイラー展開を1次までとしたが、より高次精度を実現するために、高次まで採用して用いることがある。これを高次差分式といい、有名なものにルンゲ・クッタ法がある。

\subsection{陽解法と陰解法}


\end{document}
