\documentclass[a4j,twoside,openright,11pt]{jreport}
%
\usepackage{amsmath,amssymb}
\usepackage{bm}
\usepackage{graphicx}
\usepackage{ascmac}
\usepackage{listliketab}
\usepackage{url}
\usepackage{listings}
\usepackage{color}

\setlength{\textwidth}{15.92cm}
\setlength{\oddsidemargin}{0mm}
\setlength{\evensidemargin}{0mm}
\setlength{\topmargin}{-1cm}
\setlength{\textheight}{23.5cm}
\setlength{\footskip}{18mm}

%
\pagestyle{plain}
\title{伝熱学\\第6回}
\author{九州工業大学 機械知能工学科 機械知能コース 3年\\学籍番号:13104069 坂本悠作}
\date{\today}

\begin{document}
\maketitle
\newpage

\chapter{平成25年期末試験}
\section{Describe the three fundamental modes of heat transfer and those basic laws in detail.}
\begin{enumerate}
\item 熱伝導-フーリエの法則
\item 対流伝熱-ニュートンの冷却則
\item 輻射伝熱-シュテファン・ボルツマン
\end{enumerate}

\section{}

\section{図に示す軸方向(奥行き)に長さ1mの同軸2重円筒が定常状態にある。内側にある半径$r_1$=1cmの円筒の外表面温度$T_1$=20℃,放射率$\epsilon$=0.5である。外側にある半径$r_2$=2cmの円筒の内表面温度$T_2$=200℃,放射率$\epsilon$=0.1とした場合、次のことを求めよ}
\subsection{形態係数$F_{11}$と$F_{22}$}
形態係数とは、添字1から出た電磁波が添字2にどれだけ伝達するのかを示す係数。
\begin{eqnarray}
F_{11} =0,F_{12} =1,F_{21} =\frac{A_1}{A_2},F_{22} =1-\frac{A_1}{A_2}\\
この問題において、F_{21} =\frac{1}{2}=0.5,F_{22} =0.5で正解
\end{eqnarray}

\subsection{この2重円筒間の輻射伝熱量}
無限級数の授業で、以下のことを学んだ。同心円筒の場合
\begin{eqnarray}
\dot Q_{12} =\frac{\sigma(T_1^4-T_2^4)}{\frac{1}{\epsilon_1}+\frac{1-\epsilon_2}{\epsilon_2}(\frac{A_1}{A_2})}A_1
\end{eqnarray}
よって、
\begin{eqnarray}
\dot Q_{12} =\frac{5.67 \times 10^{-8}((200+273.15)^4-(20+273.15)^4)}{\frac{1}{0.5}+\frac{1-0.1}{0.1}(\frac{0.01}{0.02})}\pi\times0.02=23.42[W]
\end{eqnarray}

\section{ガスタービンの換気ガスの熱を回収するため、隔板式向流熱交換器を用いることにする。温度160度、流量$600kg/hr$の排気ガスで流量$300kg/hr$の水を20℃から80℃まで加熱する設計要求に対して、必要な隔板の伝熱面積を求めよ。ただし、水の定圧比熱は$4.2kJ/(kg \cdot K)$、ガスの定圧比熱は$1.0kJ/(kg \cdot K)$、ガス側の熱伝達率は$40W/(m^2 \cdot K)$,水側の熱伝達率は$160W/(m^2 \cdot K)$,隔板の熱伝導率は$150W/(m \cdot K)$,厚みは$1mm$とする。}

水に与えられる熱量を求める
\begin{eqnarray}
Q_{water} &=& \dot m C_P \Delta T\\
  &=& 300 \times \frac{1}{3600} \times 4200 \times (80-20)\\
  &=& 21000[W]
\end{eqnarray}
ガスの出口温度を求める
\begin{eqnarray}
Q_{gass}   &=& \dot m C_P \Delta T = Q_{water}\\
21000     &=& 600 \times \frac{1}{3600} \times 1000 \times (160-T_{gass(out)})\\
 T_{gass(out)}        &=& 34[℃]
\end{eqnarray}
熱通過率を計算する.
\begin{eqnarray}
熱通過率 &=& \frac{1}{40} + \frac{1}{160} + \frac{0.001}{150}\\
&=&0.03126
\end{eqnarray}
対数温度平均は、以下のように算出できる
\begin{eqnarray}
\Delta T_m = \frac{80-14}{\ln \frac{80}{14}} = 37.87
\end{eqnarray}
フーリエの法則より
\begin{eqnarray}
Q &=& \frac{\Delta T_m}{k}A\\
A &=& 21000 \times 0.03126 / 37.87\\
&=& 17.33[m^2]
\end{eqnarray}

\chapter{平成23年度}
\section{図に示す2次元物体について、定常状態における内部$(1\sim 6の格子点)$の温度を数値的に求めよ。ただし、格子間隔は$\Delta x = \Delta y =1cm$で、左の面は断熱されており、右面と下面は100度に保たれている。上面では熱伝達率は$25W/(m^2K)$で、温度5度の周囲流体と熱伝達を行っているものとする。この物体の熱伝導率は、$2.5W/(mK)$である。}

\end{document}