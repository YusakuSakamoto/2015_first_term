\documentclass[a4j,twoside,openright,11pt]{jsarticle}
%
\usepackage{amsmath,amssymb}
\usepackage{bm}
\usepackage{graphicx}
\usepackage{ascmac}
\usepackage{listliketab}
\usepackage{url}
\usepackage{listings}

\setlength{\textwidth}{15.92cm}
\setlength{\oddsidemargin}{0mm}
\setlength{\evensidemargin}{0mm}
\setlength{\topmargin}{-1cm}
\setlength{\textheight}{23.5cm}
\setlength{\footskip}{18mm}

%
\pagestyle{plain}
\begin{document}

\begin{screen}
\huge
\begin{center}
{\bf 流体力学 第3回}\\
\end{center}

\normalsize
\begin{flushright}
九州工業大学 機械知能工学科 機械知能コース  坂本 悠作\\連絡先:n104069y@mail.kyutech.jp \hspace{0.2in}提出日 2015年4月22日
\end{flushright}
\end{screen}

\setcounter{section}{2} 
\setcounter{subsection}{2}
\subsection{流体の加速度}
\subsubsection{速度について}
\begin{itemize}
\item ラグランジュの方法\\
質点の速度に着目して、$x(t),y(t),z(t)$を追跡して、流体塊の運動を捉える
\item オイラーの方法\\
空間に固定された場所を考え、その場所を次々と占める流体塊の運動を捉える
\end{itemize}
今、空間中のある点$P(x,y,z,t)$について考える。
この点を追跡して、ある点に移動した点の座標は$P'(x+dx,y+dy,z+dz)$と表現できる。
これらの点の間の速さを、$\bm {V}(u,v,w)$とする。
このときの$u$について考える。\\
\begin{eqnarray}
du &=& u(x+dx,y+dy,z+dz,t+dt) - u(x,y,z,t)\\
   &=& (u+\frac{\partial u}{\partial x}dx+\frac{\partial u}{\partial y}dy+\frac{\partial u}{\partial z}dz+\frac{\partial u}{\partial t}dt)-u\\
   &=& \frac{\partial u}{\partial t}dt+\frac{\partial u}{\partial x}dx+\frac{\partial u}{\partial y}dy+\frac{\partial u}{\partial z}dz
\end{eqnarray}
方向の加速度は、
\begin{eqnarray}
\frac{du}{dt} &=& \frac{\partial u}{\partial t}+\frac{\partial u}{\partial x}\frac{dx}{dt}+\frac{\partial u}{\partial y}\frac{dy}{dt}+\frac{\partial u}{\partial z}\frac{dz}{dt}\\
              &=& \frac{\partial u}{\partial t}+u\frac{\partial u}{\partial x}+v\frac{\partial u}{\partial y}+w\frac{\partial u}{\partial z}
\end{eqnarray}
同様にしてまとめると、
\begin{eqnarray}
\frac{du}{dt} &=& \frac{\partial u}{\partial t}+u\frac{\partial u}{\partial x}+v\frac{\partial u}{\partial y}+w\frac{\partial u}{\partial z}\\
\frac{dv}{dt} &=& \frac{\partial v}{\partial t}+u\frac{\partial v}{\partial x}+v\frac{\partial v}{\partial y}+w\frac{\partial v}{\partial z}\\
\frac{dw}{dt} &=& \frac{\partial w}{\partial t}+u\frac{\partial w}{\partial x}+v\frac{\partial w}{\partial y}+w\frac{\partial w}{\partial z}
\end{eqnarray}
従って、加速度$\bm{a}$は以下のようになる
\begin{eqnarray}
\bm{a}&=&(\frac{du}{dx},\frac{dv}{dy},\frac{dw}{dz})\\
      &=&(\frac{\partial }{\partial t}+u\frac{\partial }{\partial x}+v\frac{\partial }{\partial y}+w\frac{\partial }{\partial z})\bm{V}\\
      &=&\frac{D}{Dt}\bm{V}
\end{eqnarray}
上の式で、$\frac{D}{Dt}$をラグランジュ微分(実質微分)という。
\subsection{質量力と表面力}

\subsubsection{質量力(体積力)}
重力や電磁気のように、質量や電荷を作用する力\\
$\bm{F}(x,y,z)$とすると、重力であれば$(x,y,z)=(0,0,-g)$
\subsection{応力テンソル(垂直応力)}
今、x軸y軸z軸に対して平行な面を持つ直方体を考える。\\
垂直応力は$\sigma_x,\sigma_y,\sigma_z$の3つ,剪断力は$\tau_{xy},\tau_{yx},\tau_{yz},\tau_{zy},\tau_{zx},\tau_{xz}$の6つ、合計9つの力成分をそれぞれかけることができる。\\
ここでは、特別に次のような関係式が成り立っているとすると、この行列式は対称行列になる。
\begin{eqnarray}
\tau_{xy}&=&\tau_{yx}\nonumber\\
\tau_{yz}&=&\tau_{zy}\nonumber\\
\tau_{zx}&=&\tau_{xz}\nonumber
\end{eqnarray}

\begin{equation}
\bm{T}=
\begin{pmatrix}
\sigma_x & \tau_{yx}   & \tau_{zx}\\
\tau_{xy} & \sigma_{y} &\tau_{zy}\\
\tau_{xz}& \tau_{yz}&\sigma_{z}
\end{pmatrix}
\end{equation}


\subsection*{レポート1(連続の式)}
直交座標に置いて、連続の式は以下のように示される。
\begin{equation}
\frac{\partial u}{\partial x} + \frac{\partial u}{\partial y} = 0
\end{equation}
この連続の式を、$(x,y) \Rightarrow (r,\theta)$の変換を用いて、円柱座標$(r,\theta)$の形で連続の式を求めよ。\\
\\
------------------------------------------------------------\\
提出は5月13日(水):授業前に提出\\
手書きのみ評価\\
------------------------------------------------------------\\
ヒント
\begin{itemize}
\item $v_\theta=r\dot \theta$である
\item 授業では$(u,v)$と表記していたものは、$(v_r,v_\theta)$を用いて表現できる
\begin{eqnarray}
u=v_r\cos\theta - v_\theta\sin\theta\\
v=v_r\sin\theta + v_\theta\cos\theta
\end{eqnarray}
\item ここで用いるであろう演算子を以下に示す。
\begin{eqnarray}
\frac{\partial }{\partial x}=\frac{\partial }{\partial r}\frac{\partial r}{\partial x}+\frac{\partial }{\partial \theta}\frac{\partial \theta}{\partial x}\\
\frac{\partial }{\partial y}=\frac{\partial }{\partial r}\frac{\partial r}{\partial y}+\frac{\partial }{\partial \theta}\frac{\partial \theta}{\partial y}
\end{eqnarray}
\end{itemize}

\end{document}