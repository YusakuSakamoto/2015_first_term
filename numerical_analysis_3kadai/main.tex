\documentclass[a4j,twoside,openright,11pt]{jsarticle}
%
\usepackage{amsmath,amssymb}
\usepackage{bm}
\usepackage{graphicx}
\usepackage{ascmac}
\usepackage{listliketab}
\usepackage{url}
\usepackage{listings}

\setlength{\textwidth}{15.92cm}
\setlength{\oddsidemargin}{0mm}
\setlength{\evensidemargin}{0mm}
\setlength{\topmargin}{-1cm}
\setlength{\textheight}{23.5cm}
\setlength{\footskip}{18mm}

%
\pagestyle{plain}
\begin{document}

\begin{screen}
\huge
\begin{center}
{\bf 数値解析法 第3課題}\\
\end{center}

\normalsize
\begin{flushright}
九州工業大学 機械知能工学科 機械知能コース 3年 坂本 悠作\\学籍番号 13104069 \hspace{0.2in}提出日 2015年4月28日
\end{flushright}
\end{screen}

\section*{問題}
\[
\bm{A}=
  \left[
    \begin{array}{rrr}
      1 & 1 & 1 \\
      1 & 1 & 2 \\
      1 & 2 & 1 \\
    \end{array}
  \right]
\]
上の行列の固有値、固有ベクトルを求めよ\\

\begin{itemize}
\item べき乘法
\item Jacobi法
\end{itemize}

\section{べき乘法}
べき乘法により、最大の固有値を求める。
計算すると、以下のようになる。
収束条件は以下のように設定した
\begin{equation}
\frac{\lambda^{(k-1)}}{\lambda^{(k)}}<\epsilon\;\;\;\;\;\;(\epsilon=0.005)
\end{equation}
初期設定として、固有ベクトルを以下のように仮定する
\begin{equation}
\bm{x}=
  \left(
    \begin{array}{r}
      1 \\
      1 \\
      1 \\
    \end{array}
  \right)
\end{equation}
\begin{eqnarray}
\bm{Ax}=
  \left(
    \begin{array}{rrr}
      1 & 1 & 1 \\
      1 & 1 & 2 \\
      1 & 2 & 1 \\
    \end{array}
  \right)
  \left(
    \begin{array}{r}
      1 \\
      1 \\
      1 \\
    \end{array}
  \right)&=&
  \left(
    \begin{array}{r}
      3 \\
      4 \\
      4 \\
    \end{array}
  \right)=4
  \left(
    \begin{array}{r}
      \frac{3}{4} \\
      1 \\
      1 \\
    \end{array}
  \right)
\nonumber\\
  \left(
    \begin{array}{rrr}
      1 & 1 & 1 \\
      1 & 1 & 2 \\
      1 & 2 & 1 \\
    \end{array}
  \right)
  \left(
    \begin{array}{r}
      \frac{3}{4} \\
      1 \\
      1 \\
    \end{array}
  \right)&=&
  \left(
    \begin{array}{r}
      \frac{11}{4} \\
      \frac{15}{4} \\
      \frac{15}{4} \\
    \end{array}
  \right)=\frac{15}{4}
  \left(
    \begin{array}{r}
      \frac{11}{15} \\
      1 \\
      1 \\
    \end{array}
  \right)
\nonumber
\\
  \left(
    \begin{array}{rrr}
      1 & 1 & 1 \\
      1 & 1 & 2 \\
      1 & 2 & 1 \\
    \end{array}
  \right)
  \left(
    \begin{array}{r}
      \frac{11}{15} \\
      1 \\
      1 \\
    \end{array}
  \right)&=&
  \left(
    \begin{array}{r}
      \frac{41}{15} \\
      \frac{56}{15} \\
      \frac{56}{15} \\
    \end{array}
  \right)=\frac{56}{15}
  \left(
    \begin{array}{r}
      \frac{41}{56} \\
      1 \\
      1 \\
    \end{array}
  \right)
\nonumber
\\
\nonumber\\
\nonumber\\
\frac{\lambda^{(k-1)}}{\lambda^{(k)}}の値が0.00031となったので、ここで終了とした。\nonumber
\\
  \left(
    \begin{array}{rrr}
      1 & 1 & 1 \\
      1 & 1 & 2 \\
      1 & 2 & 1 \\
    \end{array}
  \right)
  \left(
    \begin{array}{r}
      \frac{41}{56} \\
      1 \\
      1 \\
    \end{array}
  \right)=
  \left(
    \begin{array}{r}
      \frac{153}{56} \\
      \frac{209}{56} \\
      \frac{209}{56} \\
    \end{array}
  \right)&=&\frac{209}{56}
  \left(
    \begin{array}{r}
      \frac{153}{209} \\
      1 \\
      1 \\
    \end{array}
  \right) \approx 3.73214
  \left(
    \begin{array}{r}
      0.73205 \\
      1 \\
      1 \\
    \end{array}
  \right)
\end{eqnarray}
\subsubsection*{結果}
\begin{equation}
固有値\lambda = \frac{209}{56},\;\;\;固有ベクトルv=\left(0.73205,1,1\right)
\end{equation}
\section{Jacobi法}
行列$\bm{A}$は実対称行列なので、Jacobi法が使える。\\
\begin{equation}
\bm{P_1}=
\left[
\begin{array}{ccc}
  1&0&0\\
  0&\cos \theta_1 &-\sin\theta_1\\
  0&\sin \theta_1 &\cos\theta_1\\
\end{array}
\right]
\end{equation}
問題より、$a=1,b=2,d=1$であるので、$a=d$より$\theta=\pi/4$を用いる。
\begin{eqnarray}
  \bm{P_1}^{T}\bm{A}\bm{P_1} 
  &=& 
  \left[
  \begin{array}{ccc}
  1&0&0\\
  0&\cos \pi/4 &\sin \pi/4\\
  0&-\sin \pi/4 &\cos \pi/4\\
  \end{array}
  \right]
  \left[
    \begin{array}{rrr}
      1 & 1 & 1 \\
      1 & 1 & 2 \\
      1 & 2 & 1 \\
    \end{array}
  \right]
  \left[
  \begin{array}{ccc}
  1&0&0\\
  0&\cos \pi/4 &-\sin \pi/4\\
  0&\sin \pi/4 &\cos \pi/4\\
  \end{array}
  \right]
\nonumber\\
  &=&
  \left[
  \begin{array}{ccc}
  1&\sqrt{2}&0\\
  \sqrt{2}&3 &0\\
  0&0 &-1\\
  \end{array}
  \right]
\nonumber
\\
ここで、\theta &=& \frac{2\sqrt{2}}{(1-3)} \div 2 =-0.477658 (radian)であるので、
\nonumber
\\
  \bm{P_2}^{T}\bm{A}\bm{P_2} 
  &=& 
  \left[
  \begin{array}{ccc}
  \cos -0.477658 &\sin -0.477658  &0\\
  -\sin -0.477658 &\cos -0.477658 &0\\
  0&0&1\\
  \end{array}
  \right]
  \left[
  \begin{array}{ccc}
  1&\sqrt{2}&0\\
  \sqrt{2}&3 &0\\
  0&0 &-1\\
  \end{array}
  \right]
  \left[
  \begin{array}{ccc}
  \cos -0.47765 &-\sin -0.47765&0\\
  \sin -0.47765 &\cos -0.47765&0\\
  0&0&1\\
  \end{array}
  \right]
\nonumber
\\
  &=&
  \left[
  \begin{array}{ccc}
  0.267949 & 1.07062 \times 10^{-6}&0\\
  1.07062 \times 10^{-6} &3.73205&0\\
  0&0&-1\\
  \end{array}
  \right]
\end{eqnarray}
非対角成分が十分に小さい事$(\epsilon=0.005以下)$から、ここで収束したと判定する。\\
固有値は、$(0.267949,3.73205,-1)$となる\\
\\
この結果より、固有ベクトルを求めると、
\begin{eqnarray}
  P_1P_2 &=&  \left[
  \begin{array}{ccc}
  1&0&0\\
  0&\cos \pi/4 &-\sin \pi/4\\
  0&\sin \pi/4 &\cos \pi/4\\
  \end{array}
  \right]
  \left[
  \begin{array}{ccc}
  \cos -0.47765 &-\sin -0.47765&0\\
  \sin -0.47765 &\cos -0.47765&0\\
  0&0&1\\
  \end{array}
  \right]
\nonumber
\\
&=&   \left[
  \begin{array}{ccc}
0.88807   & 0.45969 & 0\\
-0.325052 & 0.62797 & -0.70711\\
-0.325052 & 0.62797 & 0.70711\\
  \end{array}
  \right]
\end{eqnarray}
第2、第3行の成分の大きさを1と調整すると、以下のようになる
\begin{eqnarray}
v_1&=&(0.73205,1,1)\nonumber\\
v_2&=&(0,-1,1)\nonumber\\
v_3&=&(-2.73949,1,1)\nonumber
\end{eqnarray}

\end{document}