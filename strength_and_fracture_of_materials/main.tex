\documentclass[a4j,twoside,openright,11pt]{jreport}
%
\usepackage{amsmath,amssymb}
\usepackage{bm}
\usepackage{graphicx}
\usepackage{ascmac}
\usepackage{listliketab}
\usepackage{url}
\usepackage{listings}

\setlength{\textwidth}{15.92cm}
\setlength{\oddsidemargin}{0mm}
\setlength{\evensidemargin}{0mm}
\setlength{\topmargin}{-1cm}
\setlength{\textheight}{23.5cm}
\setlength{\footskip}{18mm}

%
\pagestyle{plain}
\begin{document}

\begin{screen}
\huge
\begin{center}
{\bf 材料強度学 第2回}\\
\end{center}

\normalsize
\begin{flushright}
九州工業大学 機械知能工学科 機械知能コース  坂本 悠作\\連絡先:n104069y@mail.kyutech.jp \hspace{0.2in}提出日 2015年4月14日
\end{flushright}
\end{screen}

\section{変形}
\begin{itemize}
\item 弾性変形(elastic deformation)\\
設計においては考慮されている変形。
\item 塑性変形(Plastic deformation)\\
設計においては通常考慮しない変形。
\begin{itemize}
\item すべり変形(Slip)\\
通常の塑性変形。
\item 双晶変形(Twin)\\
高速or低温での変形。
\end{itemize}
\end{itemize}

\subsubsection{すべり}
最大せん断応力に最も近い特定の結晶面(滑り面)の上下が、特定の方向(すべり方向)へ相互に起こる仕組み。\\
\begin{itemize}
\item すべり系(slip system)=すべり面$\times$すべり方向
\item すべり系の数=変形のしやすさ
\end{itemize}
\begin{table}[htb]
\begin{center}
  \caption{すべり面}
\small
  \begin{tabular}{|l||c|c|c|} \hline
結晶構造&すべり面(稠密面)                       &総数   &備考\\
\hline
hcp(六方最密充填構造) &(0 0 0 1) $<$1 1 2 0$>$             &3通り  &チタン合金、マグネシウム合金等。\\
                      &                                 &        &変形しにくい\\
\hline
fcc(面心立方格子構造) &$\bigl\{$1 1 1$\bigr\}$  $<$1 1 0$>$                &12通り &加工硬化しやすい。銅やステンレス等\\
\hline
bcc(体心立方格子構造) &$\bigl\{$1 1 0$\bigr\}$ $\bigl\{$2 1 1$\bigr\}$ $\bigl\{$3 2 1$\bigr\}$ $<$1 1 1$>$ &48通り &ナトリウム、鉄。\\
&&&一度に多くのすべり系が動いてしまう\\
\hline
  \end{tabular}
\end{center}
\end{table}
\normalsize

理想的降伏強度(理想的限界すべり応力$\tau_{th}$)\\
xが最小ならば、
\begin{equation}
\tau = \tau_{th}(\frac{2 \pi x}{b})
\end{equation}
この時は、弾性変形なので
\begin{equation}
\tau = G\gamma = G\frac{x}{a}
\end{equation}
両式より
\begin{equation}
\tau_{th} = G \frac{b}{2 \pi a} \approx \frac{1}{10} G
\end{equation}

\subsubsection{参考:銅の場合}
銅の$G:8.1 \times 10^4 GPa$\\
\begin{eqnarray}
\tau_{th} &\approx& \frac{1}{10}G \approx 1\times 10^4        [MPa]\\
         &\approx& 100 \sim 1000                                [MPa]\\
\frac{\tau_{i}}{\tau_{th}} &=& \frac{1}{10} \sim \frac{1}{1000}
\end{eqnarray}

\subsection{変形に影響する因子}
格子欠陥\\
\begin{itemize}
\item 点欠陥
\begin{itemize}
\item 原子空孔
\item 格子間原子
\item 不純物原子
\end{itemize}
\item 線欠陥
\begin{itemize}
\item 転位
\end{itemize}
\item 面欠陥
\begin{itemize}
\item 積層欠陥
\item 双晶欠陥
\item 結晶粒界
\end{itemize}
\end{itemize}

\subsection{原子空孔(Vacancy)}
格子中のあるべき場所に原子が存在しない。\\
原子空孔濃度:熱力学的平衡濃度\\
\begin{equation}
\frac{x}{N} = \exp(-\frac{E_s}{kt})
\end{equation}
ここで、\\
n:空孔数\\
N:格子点数\\
$E_s$:1原子を表面移動させるのに要するエネルギ

\subsection{格子間原子(Internal atom)}
格子点以外に原子がいる状態。 中性子線によって、原子がはじき出されてできる。\\
$\Rightarrow$放射線損傷(Radiation damage) , 硬化(靭性の低下) , 延性-脆性遷移温度の上昇\\

\subsection{不純物原子( impurity atom)}
$\Rightarrow$合金\\
\begin{itemize}
\item 置換系固有原子 格子点の原子が置き換わる \\
Fe $\Leftarrow$ Mo,Ma,Ni,Cr,Co\\
Pb $\Leftarrow$ Zn\\

\item 侵入型固有原子 格子間に原子が入り込む\\
Fe $\Leftarrow$ C,N,H\\
Au $\Leftarrow$ Ag,Cu,Hg\\
\end{itemize}

\subsection{転位}
すべり面の上下で、一度に全体がすべらず、1部分ずつ移動した時の境界転位線は閉曲線(表面を含む)
\begin{itemize}
\item バーガース・ベクトル
\item 刃状転位
\item 螺旋転位
\end{itemize}


\end{document}