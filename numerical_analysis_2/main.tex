\documentclass[a4j,twoside,openright,11pt]{jreport}
%
\usepackage{amsmath,amssymb}
\usepackage{bm}
\usepackage{graphicx}
\usepackage{ascmac}
\usepackage{listliketab}
\usepackage{url}
\usepackage{listings}

\setlength{\textwidth}{15.92cm}
\setlength{\oddsidemargin}{0mm}
\setlength{\evensidemargin}{0mm}
\setlength{\topmargin}{-1cm}
\setlength{\textheight}{23.5cm}
\setlength{\footskip}{18mm}

%
\pagestyle{plain}
\begin{document}

\begin{screen}
\huge
\begin{center}
{\bf 数値解析法 第2回課題}\\
\end{center}

\normalsize
\begin{flushright}
九州工業大学 機械知能工学科 機械知能コース 3年 坂本 悠作\\学籍番号 13104069 \hspace{0.2in}提出日 2015年4月21日
\end{flushright}
\end{screen}

\section{問題}
\[
  \left[
    \begin{array}{rrrr}
      2 & -3 & -1 &0\\
      7 & 1 & 4 & 9\\
      2 & -5 & -1 & 0\\
      2 & 1 & 2 & -3\\
    \end{array}
  \right]
\left(
    \begin{array}{r}
      x_1 \\
      x_2 \\
      x_3 \\
      x_4 \\
    \end{array}
  \right)
  =
\left(
    \begin{array}{r}
      5 \\
      8 \\
      9 \\
      9 \\
    \end{array}
  \right)
\]
上の行列式において、$x_{i} (i:1,2,3,4)$の値を計算せよ

\section{直接解法}
逆行列を計算する。
\[
  \left[
    \begin{array}{rrrr|rrrr}
      2 & -3 & -1 &0  &1&0&0&0\\
      7 & 1 & 4 & 9   &0&1&0&0\\
      2 & -5 & -1 & 0 &0&0&1&0\\
      2 & 1 & 2 & -3  &0&0&0&1\\
    \end{array}
  \right]
\]
\\
逆行列は以下のように算出できた\\
\[
\frac{1}{66}
  \left[
    \begin{array}{rrrr}
      46  & 2& -26&  6\\
      33  & 0& -33&  0\\
      -73 & 4&  47& 12\\
      -7  & 4&   3&-10
    \end{array}
  \right]
\]
これを左から$\left( 5,8,9,9  \right)$と内積計算すると、解$\left( x_1,x_2,x_3,x_4 \right)$=$\left(1,-2,3,-1\right)$が得られる。
\section{反復解法}
反復解法で解を求めるにあたり、対角優位になるように作り変える。\\
次の行列の1行目は、問題の行列の1行目と4行目を足したもので、他の行は順番を入れ替えて作ったものである。

\[
  \left[
    \begin{array}{rrrr}
      4 & -2 & 1 & -3\\
      2 & -5 & -1 & 0\\
      2 & 1 & 2 & -3\\
      7 & 1 & 4 & 9\\
    \end{array}
  \right]
\left(
    \begin{array}{r}
      x_1 \\
      x_2 \\
      x_3 \\
      x_4 \\
    \end{array}
  \right)
  =
\left(
    \begin{array}{r}
      14 \\
      8 \\
      9 \\
      9 \\
    \end{array}
  \right)
\]
行列の形から方程式の形に置き換えると、以下のようになる。
\begin{eqnarray}
4x_1-2x_2+x_3-3x_4&=&14\\
2x_1-5x_2-x_3&=&9\\
2x_1+x_2+2x_3-3x_4&=&9\\
7x_1+x_2+4x_3+9x_4&=&8
\end{eqnarray}
$x_1,x_2,x_3,x_4$を求めるために、それぞれ上の式から変形させる
\begin{eqnarray}
x_1 &=& \left( 14+2x_2-x_3+3x_4 \right) /4  \\
x_2 &=& \left( 9 -2x_1+x_3      \right) /-5 \\ 
x_3 &=& \left( 9 -2x_1-x_2+3x_4 \right) /2  \\ 
x_4 &=& \left( 8 +7x_1-x_2+4x_3 \right) /9
\end{eqnarray}

今回はgauss seidel法を用いて計算する。\\
初期値は0,1,3,0とし、計算には表計算ソフトを利用した。\\
この計算により、解は$\left( x_1,x_2,x_3,x_4 \right)$=$\left(1,-2,3,-1\right)$

\end{document}