\documentclass[a4j,twoside,openright,11pt]{jsarticle}
%
\usepackage{amsmath,amssymb}
\usepackage{bm}
\usepackage{graphicx}
\usepackage{ascmac}
\usepackage{listliketab}
\usepackage{url}
\usepackage{listings}

\setlength{\textwidth}{15.92cm}
\setlength{\oddsidemargin}{0mm}
\setlength{\evensidemargin}{0mm}
\setlength{\topmargin}{-1cm}
\setlength{\textheight}{23.5cm}
\setlength{\footskip}{18mm}

%
\pagestyle{plain}
\begin{document}

\begin{screen}
\huge
\begin{center}
{\bf 弾塑性力学}\\
\end{center}

\normalsize
\begin{flushright}
九州工業大学 機械知能工学科 機械知能コース  坂本 悠作\\連絡先:n104069y@mail.kyutech.jp
\end{flushright}
\end{screen}

\section{コーシーのひずみ}
工学ひずみとも呼ばれる。初期の長さに対する変化量を表している。
\begin{eqnarray}
\gamma = \frac{\delta x}{l}
\end{eqnarray}
\section{ヘンキーの応力・ひずみ関係}
ヘンキーひずみは、対数ひずみとも呼ばれており、コーシーのひずみに比べてひずみ経路の影響を考慮して、増分変形の連続で生じた最終的なひずみを表す。
\begin{eqnarray}
\gamma = \ln \frac{\delta x}{l}
\end{eqnarray}
\section{ルイスの応力・ひずみ関係}
\section{相当ひずみ}
\section{全ひずみ理論}
\section{ひずみ増分理論}
\section{プラントル・ルイス}
\section{レビー・ミーゼス}
\end{document}
