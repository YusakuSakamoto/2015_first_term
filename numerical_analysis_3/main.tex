\documentclass[a4j,twoside,openright,11pt]{jsarticle}
%
\usepackage{amsmath,amssymb}
\usepackage{bm}
\usepackage{graphicx}
\usepackage{ascmac}
\usepackage{listliketab}
\usepackage{url}
\usepackage{listings}

\setlength{\textwidth}{15.92cm}
\setlength{\oddsidemargin}{0mm}
\setlength{\evensidemargin}{0mm}
\setlength{\topmargin}{-1cm}
\setlength{\textheight}{23.5cm}
\setlength{\footskip}{18mm}

%
\pagestyle{plain}
\begin{document}

\begin{screen}
\huge
\begin{center}
{\bf 数値解析法 第3回}\\
\end{center}

\normalsize
\begin{flushright}
九州工業大学 機械知能工学科 機械知能コース 3年 坂本 悠作\\学籍番号 13104069 \hspace{0.2in}提出日 2015年4月28日
\end{flushright}
\end{screen}


\setcounter{section}{2}
\section{固有値問題}
\subsubsection{正方行列とは}
正方行列には、次のような性質がある
\begin{itemize}
\item 行の数と列の数が等しい行列
\item 逆行列を定義できる
\item 行列式(スカラー)を定義できる
\end{itemize}

$\bm{A}$:n次正方行列
\begin{equation}
\bm{A}\bm{x}=\lambda\bm{x}
\end{equation}

\begin{enumerate}
\item n個の固有値
\item $\bm{A}$が実対象行列のとき、次の性質がある
\begin{itemize}
\item 固有値は全て実数となる
\item 異なる固有値に属する固有ベクトルは直交する
\end{itemize}
\item $\bm{A}$が対角行列$\rightarrow$対角成分が固有値
\item $\bm{M}$が正則$\rightarrow \bm{M}^{-1}\bm{A}\bm{M}$は、$\bm{A}$と同じ固有値
\item $\bm{M}^{-1}\bm{A}\bm{M} \Rightarrow \bm{D}(対角行列) \rightarrow \bm{D}の対角成分が固有値$\\
$\bm{M}$の各列が固有ベクトル
\end{enumerate}

\subsection{べき乘法}
絶対値が最大の固有値・固有ベクトルを求める方法
\subsubsection{考え方}
\begin{equation}
\bm{A}の固有値 \| \lambda_1 \| \geq \|\lambda_2\| \geq \cdots \geq\|\lambda_N\| \nonumber
\end{equation}
\begin{equation}
固有ベクトル \bm{u}_1,\bm{u}_2,\bm{u}_3,\cdots \nonumber
\end{equation}
$\bm{u}$は互いに1次独立なので、初期ベクトル$\bm{x^{(0)}}$はこれらの1次結合により、次のようになる。()内の数字は$\bm{A}$を何度かけたものかを表している。
\begin{equation}
任意のベクトル \bm{x^{(0)}}=c_1\bm{u}_1+c_2\bm{u}_2+c_3\bm{u}_3+\cdots \nonumber
\end{equation}
よって、
\begin{eqnarray}
\bm{Ax^{(0)}} &=&c_1\bm{Au}_1+c_2\bm{Au}_2 \cdots c_N\bm{Au}_N\nonumber\\
             &=&c_1\lambda_1\bm{u}_1+c_2\lambda_2\bm{u}_2 \cdots c_N\lambda_N\bm{u}_N\nonumber
\end{eqnarray}
この計算の()の値を大きくしていくと、
\begin{eqnarray}
  \bm{x^{(k)}}=\bm{A^kx^{(0)}} &=&c_1\bm{A^ku}_1+c_2\bm{A^ku}_2 \cdots c_N\bm{A^ku}_N\nonumber\\
                               &=&c_1\lambda_1^k\bm{u}_1+c_2\lambda_2^k\bm{u}_2 \cdots c_N\lambda_N^k\bm{u}_N\nonumber\\
                               &=&c_1\lambda_1^k \left\{ u_1+\frac{c_2}{c_1} \left( \frac{\lambda_2}{\lambda_1} \right)^ku_2 + \cdots + \frac{c_N}{c_1} \left( \frac{\lambda_N}{\lambda_1} \right)^ku_N \right\}
 \end{eqnarray}
仮定より、$\lambda_1$が最も大きい値であるので、$\left( \frac{\lambda_N}{\lambda_1} \right)^k$の値は、kが大きいと0に収束する。よって、第一項目以降を0と考えると、$c_1\lambda_1\bm{u_1}$となる。以上の議論により、
\[
    \lim_{k \to \infty} \frac{\bm{x^{(k)T}x^{(k)}}}{\bm{x^{(k)T}x^{(k-1)}}} = \lambda_1
\]
これにより、$\lambda_1$が求められる。
\subsubsection{例題}
\begin{equation}
\bm{A}=\left(
\begin{array}{ccc}
2&2&-1\\
0&-1&0\\
0&-5&3
\end{array}
\right),x^{(0)}=
\begin{pmatrix}
1\\
1\\
1
\end{pmatrix}
\end{equation}
このときの固有値・固有ベクトルを求めよ\\
\subsubsection{解答}
このときの最大固有値は2.9992である。
\subsection{Jacobi法}
\subsubsection{考え方}
$\bm{A}$:n次正方行列\\
n個の固有値・固有ベクトルを同時に求める
\begin{equation}
\bm{A} = \left[
\begin{array}{cc}
a & b\\
b & d
\end{array}
\right]\\
\end{equation}

\begin{equation}
\bm{p} = \left[
\begin{array}{cc}
\cos \theta & -\sin \theta \\
\sin \theta & \cos \theta
\end{array}
\right]
\end{equation}

\begin{equation}
\bm{p^{-1}} = \left[
\begin{array}{cc}
\cos \theta & \sin \theta \\
-\sin \theta & \cos \theta
\end{array}
\right]
\end{equation}

\begin{enumerate}
\item $\bm{P}$が正則$\rightarrow \bm{P}^{-1}\bm{A}\bm{P}$は、$\bm{A}$と同じ固有値
\item $\bm{P}^{-1}\bm{A}\bm{P} \rightarrow \bm{D}(対角行列) \rightarrow \bm{D}の対角成分が固有値$\\
$\bm{P}$の各列が固有ベクトル
\end{enumerate}

\begin{equation}
\bm{P}^{-1}\bm{A}\bm{P} = 
\left[
\begin{array}{cc}
\cos \theta & -\sin \theta \\
\sin \theta & \cos \theta
\end{array}
\right]
\left[
\begin{array}{cc}
a & b \\
b & d
\end{array}
\right]
\left[
\begin{array}{cc}
\cos \theta &  \sin \theta \\
-\sin \theta & \cos \theta
\end{array}
\right]=
\left[
\begin{array}{cc}
\lambda_1&0\\
0&\lambda_2
\end{array}
\right]
\end{equation}


\begin{eqnarray}
a &=&d \Rightarrow \theta =\frac{\pi}{4}\\
a &\neq& d \Rightarrow \tan 2\theta = \frac{2b}{a-d} 
\end{eqnarray}

\subsubsection{例題}
\[
\bm{A}=
  \left[
    \begin{array}{rrr}
      1  & -2 & -2 \\
      -2 & 2 & 0 \\
      -2 & 0 & 0 \\
    \end{array}
  \right]
\]
対称行列であるので、Jacobi法が使える。非対角成分の絶対値最大のものを正とする。
\subsubsection{解答}
\begin{enumerate}
\item 絶対値最大のものを見つける
\item その要素が0になるように回転行列P1を転地行列にして左からかける
\item 右からP1をかける
\item 計算して出てきたものを$A^{(1)}$とする
\item 新たに行列P2を作る
\item 繰り返し
\end{enumerate}
非対角行列の値が小さくなれば答えとする。\\
固有値4,-2,1\\
固有ベクトル$v_1=(-2,2,1),v_2=(2,1,2),v_3=(-2,-2,2)$
\section*{問題}
\[
\bm{A}=
  \left[
    \begin{array}{rrr}
      1 & 1 & 1 \\
      1 & 1 & 2 \\
      1 & 2 & 1 \\
    \end{array}
  \right]
\]
上の行列の固有値、固有ベクトルを求めよ\\

\begin{itemize}
\item べき乘法
\item Jacobi法
\item 厳密解を求めて評価すると加点
\end{itemize}



\end{document}